% resume.tex
% vim:set ft=tex spell:

\documentclass[10pt,letterpaper]{article}
\usepackage[letterpaper,margin=0.75in]{geometry}
\usepackage[utf8]{inputenc}
\usepackage{mdwlist}
\usepackage[T1]{fontenc}
\usepackage{textcomp}
\usepackage{tgpagella}
\pagestyle{empty}
\setlength{\tabcolsep}{0em}

% indentsection style, used for sections that aren't already in lists
% that need indentation to the level of all text in the document
\newenvironment{indentsection}[1]%
{\begin{list}{}%
	{\setlength{\leftmargin}{#1}}%
	\item[]%
}
{\end{list}}

% opposite of above; bump a section back toward the left margin
\newenvironment{unindentsection}[1]%
{\begin{list}{}%
	{\setlength{\leftmargin}{-0.5#1}}%
	\item[]%
}
{\end{list}}

% format two pieces of text, one left aligned and one right aligned
\newcommand{\headerrow}[2]
{\begin{tabular*}{\linewidth}{l@{\extracolsep{\fill}}r}
	#1 &
	#2 \\
\end{tabular*}}

% make "C++" look pretty when used in text by touching up the plus signs
\newcommand{\CPP}
{C\nolinebreak[4]\hspace{-.05em}\raisebox{.22ex}{\footnotesize\bf ++}}

% and the actual content starts here
\begin{document}

\begin{center}
{\LARGE \textbf{Base de Datos de Flight Trace}}

999 E Wacker Drive\ \ \textbullet
\ \ Oficina\ 1001\ \ \textbullet
\ \ Santiago, 2013
\\
(999) 999-9999\ \ \textbullet
\ \ test@example.com
\end{center}

\hrule
\vspace{-0.4em}

\subsection*{Vuelo}

\begin{itemize}
	\parskip=0.1em

	\item
	\headerrow
		{\textbf{vuelo}}
		{\textbf{Casa Andrés}}
	\\
	\headerrow
		{\emph{tabla de vuelo}}
		{\emph{03/03/2013}}
	\begin{enumerate}
		\item Es uno o más tramos que recorre un avión, entre dos o más aeropuertos.
		\item Un vuelo debe tener un aeropuerto de salida y un aeropuerto de arribo.
		\item No existen dos vuelos iguales.
		\item Para iniciar un vuelo se necesita, un avión disponible, una tripulación disponible, compuesta por una tripulación de vuelo (flight crew) y una tripulación de cabina (cabin crew).
		\item Debe tener una ruta disponible.
		\item Puede o no, tener pasajeros, si no tiene, se denomina vuelo Ferry (Existen vuelos de Pasajeros, Carga, Ferry y combinaciones de estos).
		\item Puede o no, llevar carga o correo.
		\item Un vuelo debe tener una fecha con hora de inicio y una fecha con hora de llegada.
		\item Cuando el vuelo termina se produce un Resumen de Vuelo.
		\item Un solo vuelo puede tener más de una Ruta, esto quiere decir que hay pasajeros que se bajan en aeropuertos intermedios.
	\end{enumerate}

\end{itemize}

\subsection*{Aeropuerto}

\begin{itemize}
	\parskip=0.1em

	\item
	\headerrow
		{\textbf{aeropuerto}}
		{\textbf{Casa Andrés}}
	\\
	\headerrow
		{\emph{tabla de aeropuerto}}
		{\emph{03/03/2013}}
	\begin{enumerate}
		\item Lugar al cual llegan o salen aviones.
		\item Puede o no tener una base de la empresa.
		\item Si no hay una base de la empresa, puede o no existir handling de terceros: si existe, entonces se contratan los servicios externos de "handling" (servicios handling de terceros), lo que tiene un costo asociado; si no existe, se lleva personal en el vuelo al destino.
		\item Tiene un nombre.
		\item Tiene un código IATA (International Air Transport Association).
		\item Tiene un código ICAO (International Civil Aviation Organization).
		\item Tiene un País, Ciudad, Región (Comuna).
		\item Tiene una Calle y Numero.
		\item Tiene una casilla de correos.
		\item Tiene uno o mas teléfonos.
		\item Tiene uno o mas Fax.
		\item Tiene uno o mas e-mail.
		\item Tiene un sito web.
		\item Puede o no tener Policía Internacional disponible, Servicio de Aduana disponible y Seguridad Aeroportuaria disponible.
		\item Tiene una Taza de Embarque.
		\item Tiene un horario de atención.
		\item Tiene una o mas Pistas.
		\item Tiene un tipo de categoría de operación de aeropuerto.
		\item Tiene una categoría SEI (Servicio de Extinción de Incendios).
		\item Tiene Coordenadas GPS asociadas.
		\item Tiene o no combustible disponible.
		\item tiene o no horario de carga de combustible.
		\item Tiene Restricciones.
	\end{enumerate}

\end{itemize}

\subsection*{Avión}

\begin{itemize}
	\parskip=0.1em

	\item
	\headerrow
		{\textbf{avión}}
		{\textbf{Casa Andrés}}
	\\
	\headerrow
		{\emph{tabla de avión}}
		{\emph{07/03/2013}}
	\begin{enumerate}
		\item Es un medio de transporte que lleva a pasajeros y/o carga, desde un Aeropuerto a otro.
		\item Tiene un estado de disponible o no disponible.
		\item Tiene un registro (matricula).
		\item Tiene un versión (ejemplo 3/4).
		\item Tiene una configuración (cantidad de pasajeros).
		\item Tiene un peso máximo Taxi Weight.
		\item Tiene un peso máximo Takeoff Weight.
		\item Tiene un peso máximo Landing Weight.
		\item Tiene un peso máximo Zero Fuel Weight.
		\item Tiene un peso máximo Payload.
		\item Tiene un máximo Seating Capacity.
		\item Tiene una capacidad máxima Cargo Volume.
		\item Tiene una capacidad máxima Cargo Weight.
		\item Tiene un máximo de capacidad de combustible.
		\item Tiene un mínimo de combustible para operación en tierra.
		\item Tiene un Basic Weight.
		\item Tiene un peso de Items Operacionales.
		\item Tiene un Dry Operating Weight.		
		\item Tiene un Dry Operating Index.
		\item Tiene un máximo Range.
		\item Tiene un máximo Tiempo de Vuelo.
		\item Tiene un máximo Service Ceilling.
		\item Tiene una cantidad de motores.
		\item Tiene un máximo Thrust.		
		\item Tiene un máximo Speed.
		\item Tiene una Velocidad de Crucero.
		\item Tiene una o más Cabinas.
		\item Tiene una o más Bodegas.
		\item Tiene una Tripulación
	\end{enumerate}

\end{itemize}

\subsection*{Pista}

\begin{itemize}
	\parskip=0.1em

	\item
	\headerrow
		{\textbf{pista}}
		{\textbf{Casa Andrés}}
	\\
	\headerrow
		{\emph{tabla de pista}}
		{\emph{16/03/2013}}
	\begin{enumerate}
		\item Es un lugar físico perteneciente al Aeropuerto donde despegan o aterrizan Aviones.
		\item Cada Pista tiene 2 orientaciones.
		
	\end{enumerate}

\end{itemize}

\subsection*{Orientación Pista}

\begin{itemize}
	\parskip=0.1em

	\item
	\headerrow
		{\textbf{orientación pista}}
		{\textbf{Casa Andrés}}
	\\
	\headerrow
		{\emph{tabla de orientación pista}}
		{\emph{16/03/2013}}
	\begin{enumerate}
		\item Es la forma de como un avión Despega o Aterriza, con respecto a una pista.
		\item Tiene un nombre.
		\item Tiene un RWY NR (Runway Number).
		\item Tiene un BRG GEO (Geographic Bearing).
		\item Tiene un BRG MAG (Magnetic Bearing).
		\item Tiene un LEN RWY (Lenght Runway).
		\item Tiene un WID RWY (Width Runway).
		\item Tiene un PCN RSTG.
		\item Tiene un COORD THR.
		\item Tiene un ELEV TDZ RWY.
		\item Tiene un PEND.
		\item Tiene un SWY.
		\item Tiene un CWY.
		\item Tiene un STRIP.
		\item Tiene un OFZ.
		\item Tiene un OBS RMK.
	\end{enumerate}

\end{itemize}

\subsection*{Ayuda Navegacional}

\begin{itemize}
	\parskip=0.1em

	\item
	\headerrow
		{\textbf{Ayuda Navegacional}}
		{\textbf{Casa Andrés}}
	\\
	\headerrow
		{\emph{tabla de Ayuda Navegacional}}
		{\emph{16/03/2013}}
	\begin{enumerate}
		\item Cualquier dispositivo visual o electrónico, aéreo o en la superficie terrestre el cual provee información posicional y guia punto a punto al piloto y su aeronave.
		\item Tiene un tipo de dispositivo \{Nondirectional Radio Beacon NDB (antena); Tactical Air Navigation (TACAN); Distance Measuring Equipment (DME); Navigational Aid (NAVAID), GPS\}.
		\item Tiene una Ubicación Geográfica o GPS.
		
	\end{enumerate}

\end{itemize}

\subsection*{Vía Aérea}

\begin{itemize}
	\parskip=0.1em

	\item
	\headerrow
		{\textbf{Vía Aérea}}
		{\textbf{Casa Andrés}}
	\\
	\headerrow
		{\emph{tabla de Vía Aérea}}
		{\emph{16/03/2013}}
	\begin{enumerate}
		\item Camino que recorre un avión entre 2 Ayuda Navegacionales.
		
	\end{enumerate}

\end{itemize}

\subsection*{Tramo}

\begin{itemize}
	\parskip=0.1em

	\item
	\headerrow
		{\textbf{Tramo}}
		{\textbf{Casa Andrés}}
	\\
	\headerrow
		{\emph{tabla de Tramo}}
		{\emph{16/03/2013}}
	\begin{enumerate}
		\item Camino que recorre un avión entre 2 Aeropuertos.
		\item Esta compuesto de un Aeropuerto de Salida.
		\item Esta compuesto de un Aeropuerto de Llegada.
		\item Esta compuesto de Vías Aéreas entre Ayudas Navegacionales.
		
	\end{enumerate}

\end{itemize}


\subsection*{Ruta}

\begin{itemize}
	\parskip=0.1em

	\item
	\headerrow
		{\textbf{ruta}}
		{\textbf{Casa Andrés}}
	\\
	\headerrow
		{\emph{tabla de ruta}}
		{\emph{16/03/2013}}
	\begin{enumerate}
		\item El tramo definido por la Aerolínea que recorre un avión desde que Sale (Departure), hasta que Arriba en su ultimo destino programado (Arrival).
		\item Debe tener dos o más aeropuertos, uno de Salida (Departure) y varios de llegada (Arrival).
		\item Esta compuesto por uno o varios Tramos.
		
	\end{enumerate}

\end{itemize}


\subsection*{Empleado}

\begin{itemize}
	\parskip=0.1em

	\item
	\headerrow
		{\textbf{empleado}}
		{\textbf{Casa Andrés}}
	\\
	\headerrow
		{\emph{tabla de empleado}}
		{\emph{16/03/2013}}
	\begin{enumerate}
		\item Tiene un estado de Empleado.
		\item Tiene Numero de Empleado de la Empresa.
		\item Tiene Nombres y Apellidos.
		\item Tiene un Sexo.
		\item Tiene un Rut.
		\item Tiene una Nacionalidad.
		\item Tiene una Fecha de Nacimiento.
		\item Tiene una o más Direcciones.
		\item Tiene uno o más Teléfonos.
		\item Tiene un Correo Electrónico.
		\item Tiene una Licencia.
		\item Tiene un Pasaporte.
		\item Tiene uno o más Cargos dentro de la Empresa.
		\item Tiene un Horario de Trabajo.
		\item Tiene o no un Horas de Vuelo.
		
	\end{enumerate}

\end{itemize}


\subsection*{Cargo Empresa}

\begin{itemize}
	\parskip=0.1em

	\item
	\headerrow
		{\textbf{Cargo}}
		{\textbf{Casa Andrés}}
	\\
	\headerrow
		{\emph{El titulo de un oficio que desempeña un empleado}}
		{\emph{16/03/2013}}
	\begin{enumerate}
		\item Tiene un Nombre.
		\item Tiene una Descripción.
		\item Tiene uno o más Deberes.
		\item Tiene uno o más Derechos.
	\end{enumerate}

\end{itemize}

\subsection*{Horario de Vuelo}

\begin{itemize}
	\parskip=0.1em

	\item
	\headerrow
		{\textbf{Horario de Vuelo}}
		{\textbf{Casa Andrés}}
	\\
	\headerrow
		{\emph{Tiempo de trabajo para Empleados que tengan cargos dentro de la Tripulación}}
		{\emph{16/03/2013}}
	\begin{enumerate}
		\item ...
	\end{enumerate}

\end{itemize}

\subsection*{Horario de Trabajo}

\begin{itemize}
	\parskip=0.1em

	\item
	\headerrow
		{\textbf{Horario de Trabajo}}
		{\textbf{Casa Andrés}}
	\\
	\headerrow
		{\emph{Tiempo de trabajo para Empleados}}
		{\emph{16/03/2013}}
	\begin{enumerate}
		\item Tiene una lista de diferentes Roll de trabajo.
		\item Tiene un numero máximo de Roll.
	\end{enumerate}

\end{itemize}

\subsection*{Roll de Trabajo}

\begin{itemize}
	\parskip=0.1em

	\item
	\headerrow
		{\textbf{Roll de Trabajo}}
		{\textbf{Casa Andrés}}
	\\
	\headerrow
		{\emph{Unidad de trabajo}}
		{\emph{16/03/2013}}
	\begin{enumerate}
		\item Tiene un día.
		\item Tiene un numero que va desde 1 a 24 (que representa la hora).
	\end{enumerate}

\end{itemize}

\subsection*{Cliente}

\begin{itemize}
	\parskip=0.1em

	\item
	\headerrow
		{\textbf{Cliente}}
		{\textbf{Casa Andrés}}
	\\
	\headerrow
		{\emph{Un cliente registrado en nuestro sistema}}
		{\emph{16/03/2013}}
	\begin{enumerate}
		\item Tiene Nombre y Apellidos.
		\item Tiene un Sexo.		
		\item Tiene Rut.
		\item Tiene una Nacionalidad.
		\item Tiene Pasaporte.
		\item Tiene una Fecha de Nacimiento.
		\item Tiene una o más Direcciones.
		\item Tiene uno o más Teléfonos.
		\item Tiene un Correo Electrónico.
		\item Tiene Estadísticas de Cliente (características entregadas por la empresa, tales como puntos, descuentos o tipos de pasajeros, ya sea frecuente o no).
	\end{enumerate}

\end{itemize}

\subsection*{Pasajero}

\begin{itemize}
	\parskip=0.1em

	\item
	\headerrow
		{\textbf{pasajero}}
		{\textbf{Casa Andrés}}
	\\
	\headerrow
		{\emph{Característica de un cliente una vez que reserva un espacio en un vuelo}}
		{\emph{16/03/2013}}
	\begin{enumerate}
		\item Esta asociado a un solo cliente.
		\item Tiene un Estado.
		\item Puede o no tener Equipaje.
		\item Puede o no tener Equipaje de Mano.
		\item Tiene uno o varios Objetos Especiales o Restringidos (tales como armas, órganos en hielo seco u objetos especiales).
		\item Tiene uno o varios Tipo de Pasajero (características orgánicas de la persona, tales como peso o discapacidades especiales).
		
	\end{enumerate}

\end{itemize}

\subsection*{Equipaje}

\begin{itemize}
	\parskip=0.1em

	\item
	\headerrow
		{\textbf{Equipaje}}
		{\textbf{Casa Andrés}}
	\\
	\headerrow
		{\emph{Carga declarada de un Pasajero}}
		{\emph{16/03/2013}}
	\begin{enumerate}
		\item Tiene un Peso.
		\item Tiene un Volumen.
		\item Tiene un Destino.
	\end{enumerate}

\end{itemize}


\subsection*{Equipaje de Mano}

\begin{itemize}
	\parskip=0.1em

	\item
	\headerrow
		{\textbf{Equipaje de Mano}}
		{\textbf{Casa Andrés}}
	\\
	\headerrow
		{\emph{Carga declarada de un Pasajero}}
		{\emph{16/03/2013}}
	\begin{enumerate}
		\item Tiene un Peso.
	\end{enumerate}

\end{itemize}

\subsection*{Cargamento}

\begin{itemize}
	\parskip=0.1em

	\item
	\headerrow
		{\textbf{Cargamento}}
		{\textbf{Casa Andrés}}
	\\
	\headerrow
		{\emph{Carga de la Empresa o terceros}}
		{\emph{16/03/2013}}
	\begin{enumerate}
		\item Tiene un Peso.
		\item Tiene un Volumen.
		\item Tiene un Destino.
	\end{enumerate}

\end{itemize}

\subsection*{Tipo de Pasajero}

\begin{itemize}
	\parskip=0.1em

	\item
	\headerrow
		{\textbf{tipo de pasajero}}
		{\textbf{Casa Andrés}}
	\\
	\headerrow
		{\emph{tabla de tipo de pasajero}}
		{\emph{16/03/2013}}
	\begin{enumerate}
		\item Tiene una Condición (ya sea una discapacidad o una condición de fragilidad, como un embarazo, o traer uno o más bebes).
		\item Tiene una Descripción dada por el tipo de condición o cada Tipo de Condición.
		\item Puede o no tener asociado uno o más Implemento Extra (ejemplos: silla de ruedas, silla para bebes o coche)(lo que tienen ellos).
		\item Puede o no contratar Implementos Extra de la Empresa (ejemplos: silla de ruedas)(lo que tenemos nosotros).
	\end{enumerate}

\end{itemize}


\subsection*{Implementos Extra}

\begin{itemize}
	\parskip=0.1em

	\item
	\headerrow
		{\textbf{implementos extra}}
		{\textbf{Casa Andrés}}
	\\
	\headerrow
		{\emph{Objetos relacionados a las necesidades especiales de los pasajeros}}
		{\emph{16/03/2013}}
	\begin{enumerate}
		\item Tiene un Nombre.
		\item Tiene un Código.
		\item Tiene un Peso.
		
	\end{enumerate}

\end{itemize}


\subsection*{Objetos Especiales o Restringidos}

\begin{itemize}
	\parskip=0.1em

	\item
	\headerrow
		{\textbf{Objetos Especiales o Restringidos}}
		{\textbf{Casa Andrés}}
	\\
	\headerrow
		{\emph{Cargas Especiales de los pasajeros que no pasan como equipaje}}
		{\emph{16/03/2013}}
	\begin{enumerate}
		\item ...
	\end{enumerate}

\end{itemize}

\subsection*{Estadísticas de Cliente}

\begin{itemize}
	\parskip=0.1em

	\item
	\headerrow
		{\textbf{Estadísticas de Cliente}}
		{\textbf{Casa Andrés}}
	\\
	\headerrow
		{\emph{Tabla de información que le provee la Empresa al Cliente, como seguimiento o regalías}}
		{\emph{16/03/2013}}
	\begin{enumerate}
		\item ...
	\end{enumerate}

\end{itemize}

\subsection*{Licencia}

\begin{itemize}
	\parskip=0.1em

	\item
	\headerrow
		{\textbf{Licencia}}
		{\textbf{Casa Andrés}}
	\\
	\headerrow
		{\emph{Documento otorgado por una entidad aeronáutica o similar a un Trabajador}}
		{\emph{16/03/2013}}
	\begin{enumerate}
		\item Tiene una Validez.
		\item ...
	\end{enumerate}

\end{itemize}


\subsection*{Pasaporte}

\begin{itemize}
	\parskip=0.1em

	\item
	\headerrow
		{\textbf{Pasaporte}}
		{\textbf{Casa Andrés}}
	\\
	\headerrow
		{\emph{Documento en que consta la identidad de una persona, necesario para viajar por algunos países}}
		{\emph{16/03/2013}}
	\begin{enumerate}
		\item ...
	\end{enumerate}

\end{itemize}

\subsection*{Tripulación}

\begin{itemize}
	\parskip=0.1em

	\item
	\headerrow
		{\textbf{Tripulación}}
		{\textbf{Casa Andrés}}
	\\
	\headerrow
		{\emph{Empleados asociados a un Vuelo}}
		{\emph{16/03/2013}}
	\begin{enumerate}
		\item Tiene un Flight Crew.
		\item Tiene un Cabin Crew.
	\end{enumerate}

\end{itemize}

\subsection*{Flight Crew}

\begin{itemize}
	\parskip=0.1em

	\item
	\headerrow
		{\textbf{Flight Crew}}
		{\textbf{Casa Andrés}}
	\\
	\headerrow
		{\emph{Tripulación de Vuelo}}
		{\emph{16/03/2013}}
	\begin{enumerate}
		\item Tiene como mínimo un Empleado de Cargo Auxiliar de vuelo por cada grupo de 50 asientos(se espera que por lo menos se tenga el mínimo más uno).
		\item
	\end{enumerate}

\end{itemize}

\subsection*{Cabin Crew}

\begin{itemize}
	\parskip=0.1em

	\item
	\headerrow
		{\textbf{Cabin Crew}}
		{\textbf{Casa Andrés}}
	\\
	\headerrow
		{\emph{Tripulación de Cabina}}
		{\emph{16/03/2013}}
	\begin{enumerate}
		\item Tiene un Empleado del Cargo Capitán.
		\item Tiene un Empleado del Cargo Primer Oficial.
		\item Puede o no tener un Empleado del Cargo {Ingeniero de vuelo, Mecánico, Encargado de Vuelo, Instructor, (faltan algunos)}(Seleccionar solo uno de estos).
	\end{enumerate}

\end{itemize}

\subsection*{Resumen de Vuelo}

\begin{itemize}
	\parskip=0.1em

	\item
	\headerrow
		{\textbf{Resumen de Vuelo}}
		{\textbf{Casa Andrés}}
	\\
	\headerrow
		{\emph{Documento que se emite al finalizar un vuelo}}
		{\emph{16/03/2013}}
	\begin{enumerate}
		\item ...
	\end{enumerate}

\end{itemize}

\subsection*{Lugar de Trabajo}

\begin{itemize}
	\parskip=0.1em

	\item
	\headerrow
		{\textbf{Lugar de Trabajo}}
		{\textbf{Casa Andrés}}
	\\
	\headerrow
		{\emph{Lugar físico donde un empleado cumple funciones de su trabajo}}
		{\emph{16/03/2013}}
	\begin{enumerate}
		\item Tiene una Oficina o Base (una de las dos).
	\end{enumerate}

\end{itemize}

\subsection*{Oficina}

\begin{itemize}
	\parskip=0.1em

	\item
	\headerrow
		{\textbf{Oficina}}
		{\textbf{Casa Andrés}}
	\\
	\headerrow
		{\emph{Lugar de Trabajo donde se hacen operaciones de control de la Empresa}}
		{\emph{16/03/2013}}
	\begin{enumerate}
		\item ...
	\end{enumerate}

\end{itemize}

\subsection*{Base}

\begin{itemize}
	\parskip=0.1em

	\item
	\headerrow
		{\textbf{Base}}
		{\textbf{Casa Andrés}}
	\\
	\headerrow
		{\emph{Lugar de Trabajo donde se hacen operaciones de control de la Empresa}}
		{\emph{16/03/2013}}
	\begin{enumerate}
		\item ...
	\end{enumerate}

\end{itemize}

\subsection*{Handling}

\begin{itemize}
	\parskip=0.1em

	\item
	\headerrow
		{\textbf{Handling}}
		{\textbf{Casa Andrés}}
	\\
	\headerrow
		{\emph{Servicios externos en los aeropuertos en donde la empresa no tiene atención propia}}
		{\emph{16/03/2013}}
	\begin{enumerate}
		\item ...
	\end{enumerate}

\end{itemize}

\subsection*{Horario de Atención}

\begin{itemize}
	\parskip=0.1em

	\item
	\headerrow
		{\textbf{Horario de Atención}}
		{\textbf{Casa Andrés}}
	\\
	\headerrow
		{\emph{Es tiempo en que se atienden clientes}}
		{\emph{16/03/2013}}
	\begin{enumerate}
		\item ...
	\end{enumerate}

\end{itemize}
	
\end{document}
