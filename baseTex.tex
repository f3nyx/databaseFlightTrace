% resume.tex
% vim:set ft=tex spell:

\documentclass[10pt,letterpaper]{article}
\usepackage[letterpaper,margin=0.75in]{geometry}
\usepackage[utf8]{inputenc}
\usepackage{mdwlist}
\usepackage[T1]{fontenc}
\usepackage{textcomp}
\usepackage{tgpagella}
\pagestyle{empty}
\setlength{\tabcolsep}{0em}

% indentsection style, used for sections that aren't already in lists
% that need indentation to the level of all text in the document
\newenvironment{indentsection}[1]%
{\begin{list}{}%
	{\setlength{\leftmargin}{#1}}%
	\item[]%
}
{\end{list}}

% opposite of above; bump a section back toward the left margin
\newenvironment{unindentsection}[1]%
{\begin{list}{}%
	{\setlength{\leftmargin}{-0.5#1}}%
	\item[]%
}
{\end{list}}

% format two pieces of text, one left aligned and one right aligned
\newcommand{\headerrow}[2]
{\begin{tabular*}{\linewidth}{l@{\extracolsep{\fill}}r}
	#1 &
	#2 \\
\end{tabular*}}

% make "C++" look pretty when used in text by touching up the plus signs
\newcommand{\CPP}
{C\nolinebreak[4]\hspace{-.05em}\raisebox{.22ex}{\footnotesize\bf ++}}

% and the actual content starts here
\begin{document}

\begin{center}
{\LARGE \textbf{Base de Datos de Flight Trace}}

999 E Wacker Drive\ \ \textbullet
\ \ Oficina\ 1001\ \ \textbullet
\ \ Santiago, 2013
\\
(999) 999-9999\ \ \textbullet
\ \ test@example.com
\end{center}

\hrule
\vspace{-0.4em}

\subsection*{Vuelo}

\begin{itemize}
	\parskip=0.1em

	\item
	\headerrow
		{\textbf{vuelo}}
		{\textbf{Casa Andrés}}
	\\
	\headerrow
		{\emph{tabla de vuelo}}
		{\emph{03/03/2013}}
	\begin{enumerate}
		\item Es uno o más tramos que recorre un avión, entre dos o más aeropuertos.
		\item Un vuelo debe tener un aeropuerto de salida y un aeropuerto de arribo.
		\item No existen dos vuelos iguales.
		\item Para iniciar un vuelo se necesita, un avión disponible, una tripulación disponible, compuesta por una tripulación de vuelo (flight crew) y una tripulación de cabina (cabin crew).
		\item Debe tener una ruta disponible.
		\item Puede o no, tener pasajeros, si no tiene, se denomina vuelo Ferry.
		\item Puede o no, llevar carga o correo.
		\item Un vuelo debe tener una fecha con hora de inicio y una fecha con hora de llegada.
		\item Cuando el vuelo termina se produce un Resumen de Vuelo.
	\end{enumerate}

\end{itemize}

\subsection*{Aeropuerto}

\begin{itemize}
	\parskip=0.1em

	\item
	\headerrow
		{\textbf{aeropuerto}}
		{\textbf{Casa Andrés}}
	\\
	\headerrow
		{\emph{tabla de aeropuerto}}
		{\emph{03/03/2013}}
	\begin{enumerate}
		\item Lugar al cual llegan o salen aviones.
		\item Puede o no tener una base de la empresa.
		\item Si no hay una base de la empresa, puede o no existir handling de terceros: si existe, entonces se contratan los servicios externos de "handling" (servicios handling de terceros), lo que tiene un costo asociado; si no existe, se lleva personal en el vuelo al destino.
		\item Tiene un nombre.
		\item Tiene un código IATA (International Air Transport Association).
		\item Tiene un código ICAO (International Civil Aviation Organization).
		\item Tiene un País, Ciudad, Región (Comuna).
		\item Tiene una Calle y Numero.
		\item Tiene una casilla de correos.
		\item Tiene uno o mas teléfonos.
		\item Tiene uno o mas Fax.
		\item Tiene uno o mas e-mail.
		\item Tiene un sito web.
		\item Puede o no tener Policía Internacional disponible, Servicio de Aduana disponible y Seguridad Aeroportuaria disponible.
		\item Tiene una Taza de Embarque.
		\item Tiene un horario de atención.
		\item Tiene una o mas Pistas.
		\item Tiene un tipo de categoría de operación de aeropuerto.
		\item Tiene una categoría SEI (Servicio de Extinción de Incendios).
		\item Tiene Coordenadas GPS asociadas.
		\item Tiene restricciones.
	\end{enumerate}

\end{itemize}

\subsection*{Avión}

\begin{itemize}
	\parskip=0.1em

	\item
	\headerrow
		{\textbf{avión}}
		{\textbf{Casa Andrés}}
	\\
	\headerrow
		{\emph{tabla de avión}}
		{\emph{07/03/2013}}
	\begin{enumerate}
		\item Es un medio de transporte que lleva a pasajeros y tripulación, desde un Aeropuerto a otro.
		\item Tiene un estado de disponible o no disponible.
		\item Tiene un registro (matricula).
		\item Tiene un versión (ejemplo 3/4).
		\item Tiene una configuración (cantidad de pasajeros).
		\item Tiene un peso máximo Taxi Weight.
		\item Tiene un peso máximo Takeoff Weight.
		\item Tiene un peso máximo Landing Weight.
		\item Tiene un peso máximo Zero Fuel Weight.
		\item Tiene un peso máximo Payload.
		\item Tiene un peso máximo Seating Capacity.
		\item Tiene una capacidad máxima Cargo Volume.
		\item Tiene una capacidad máxima Cargo Weight.
		\item Tiene un máximo de capacidad de combustible.
		\item Tiene un mínimo de combustible para operación en tierra.
		\item Tiene un Basic Weight.
		\item Tiene un peso de Items Operacionales.
		\item Tiene un Dry Operating Weight.		
		\item Tiene un Dry Operating Index.
		\item Tiene un máximo Range.
		\item Tiene un máximo Tiempo de Vuelo.
		\item Tiene un máximo Service Ceilling.
		\item Tiene una cantidad de motores.
		\item Tiene un máximo Thrust.		
		\item Tiene un máximo Speed.
		\item Tiene una Velocidad de Crucero.
		
		
	\end{enumerate}

\end{itemize}

\subsection*{Ruta}

\begin{itemize}
	\parskip=0.1em

	\item
	\headerrow
		{\textbf{ruta}}
		{\textbf{Casa Andrés}}
	\\
	\headerrow
		{\emph{tabla de r}}
		{\emph{07/03/2013}}
	\begin{enumerate}
		\item El tramo que recorre un avión desde que Sale (Departure), hasta que Arriba a destino (Arrival).
		\item Debe tener dos o más aeropuertos.
		\item Esta compuesto por una ruta aérea preestablecida.
		
	\end{enumerate}

\end{itemize}

\subsection*{Molde(Tabla inexistente)}

\begin{itemize}
	\parskip=0.1em

	\item
	\headerrow
		{\textbf{titulo}}
		{\textbf{Lugar de desarrollo}}
	\\
	\headerrow
		{\emph{eslogan del titulo}}
		{\emph{Fecha}}
	\begin{enumerate}
		\item Lorem ipsum dolor sit amet, consectetuer adipiscing elit, sed
		diam nonummy nibh euismod tincidunt ut laoreet dolore magna aliquam
		erat volutpat.
	\end{enumerate}

\end{itemize}

\end{document}
